\part{Lecture 04: Monte Carlo Methods}
\title[RL Lecture 04]{Lecture 04: Monte Carlo Methods}
\date{}
\frame{\titlepage}
\frame{\frametitle{Table of contents}\tableofcontents}

%%%%%%%%%%%%%%%%%%%%%%%%%%%%%%%%%%%%%%%%%%%%%%%%%%%%%%%%%%%%%%%%%%
\section{General idea and differences to dynamic programming}
%%%%%%%%%%%%%%%%%%%%%%%%%%%%%%%%%%%%%%%%%%%%%%%%%%%%%%%%%%%%%%%%%%

%%%%%%%%%%%%%%%%%%%%%%%%%%%%%%%%%%%%%%%%%%%%%%%%%%%%%%%%%%%%%
%% Monte Carlo Methods vs. Dynamic Programming %%
%%%%%%%%%%%%%%%%%%%%%%%%%%%%%%%%%%%%%%%%%%%%%%%%%%%%%%%%%%%%%
\frame{\frametitle{Monte Carlo methods vs. dynamic programming}
  Dynamic programming:
  \begin{itemize}
  \item \hl{Model-based} prediction and control
  \item Planning inside \hl{known MDPs}
  \end{itemize}
  \vspace{1cm}
  \pause
  Monte Carlo methods:
  \begin{itemize}
  \item \hl{Model-free} prediction and control
  \item Estimating value functions and optimize policies in \hl{unknown MDPs} \pause
  \item But: still assuming finite MDP problems (or problems close to that) \pause
  \item In general: broad class of computational algorithms relying on \hl{repeated random sampling} to obtain numerical results
  \end{itemize}
}

%%%%%%%%%%%%%%%%%%%%%%%%%%%%%%%%%%%%%%%%%%%%%%%%%%%%%%%%%%%%%
%% General Monte Carlo Methods' Characteristics %%
%%%%%%%%%%%%%%%%%%%%%%%%%%%%%%%%%%%%%%%%%%%%%%%%%%%%%%%%%%%%%
\frame{\frametitle{General Monte Carlo (MC) methods' characteristics}
  \begin{itemize}
    \onslide<1->\item \hl{Learning  from experience}, i.e., sequences of samples $\left\langle x_k, u_k, r_{k+1}\right\rangle$
    \onslide<2->\item Main concept: Estimation by \hl{averaging sample returns}
    \onslide<3->\item To guarantee well-defined returns: \hl{limited to episodic tasks}
    \onslide<4->\item Consequence: Estimation and policy updates only possible in an episode-by-episode way compared to step-by-step (online)
  \end{itemize}
  \onslide<1->\begin{figure}
  \includegraphics[width=8cm]{fig/lec04/Monte_Carlo_City.jpg}
  \caption{Monte Carlo port \\(source: \href{https://www.flickr.com/photos/flynn_nrg/40890124713/}{www.flickr.com}, by \href{https://www.flickr.com/photos/flynn_nrg/}{Miguel Mendez} \href{https://creativecommons.org/licenses/by/2.0/}{CC BY 2.0})}
  \label{fig:MMonte_Carlo_City}
  \end{figure}
}

%%%%%%%%%%%%%%%%%%%%%%%%%%%%%%%%%%%%%%%%%%%%%%%%%%%%%%%%%%%%%%%%%%
\section{Basic Monte Carlo prediction}
%%%%%%%%%%%%%%%%%%%%%%%%%%%%%%%%%%%%%%%%%%%%%%%%%%%%%%%%%%%%%%%%%%
\begin{frame}
  \frametitle{Table of contents}
  \tableofcontents[currentsection]
\end{frame}

%%%%%%%%%%%%%%%%%%%%%%%%%%%%%%%%%%%%%%%%%%%%%%%%%%%%%%%%%%%%%
%% Task Description and Basic Solution %%
%%%%%%%%%%%%%%%%%%%%%%%%%%%%%%%%%%%%%%%%%%%%%%%%%%%%%%%%%%%%%
\frame{\frametitle{Task description and basic solution}
  \begin{block}{MC prediction problem statement}
    \begin{itemize}
    \item Estimate state value $v_{\pi}(x)$ for a given policy $\pi$.
    \item Available are samples $\left\langle x_{k,j}, u_{k,j}, r_{k+1,j}\right\rangle$ for episodes  $j=1,\ldots,J$.
    \end{itemize}
  \end{block}
  \pause
  \vspace{0.25cm}
  MC solution approach:
  \begin{itemize}
  \item Average returns after visiting state $x_k$ over episodes $j=1,\ldots$
    \begin{equation}
      v_{\pi}(x_k) \approx \hat{v}_{\pi}(x_k)=\frac{1}{J}\sum_{j=1}^J g_{k,j}=\frac{1}{J}\sum_{j=1}^J\sum_{i=0}^{T_j} \gamma^i r_{k+i+1,j}\, .
      \label{eq:MC_average_v_basic}
    \end{equation}
  \item Above, $T_j$ denotes the \hl{terminating time step} of each episode $j$.\pause
  \item \hl{First-visit MC}: Apply \eqref{eq:MC_average_v_basic} only to the first state visit per episode.\pause
  \item \hl{Every-visit MC}: Apply \eqref{eq:MC_average_v_basic} each time visiting a certain state per episode (if a state is visited more than one time per episode).
  \end{itemize}
}

%%%%%%%%%%%%%%%%%%%%%%%%%%%%%%%%%%%%%%%%%%%%%%%%%%%%%%%%%%%%%
%% Algorithmic Implementation:MC-Based Prediction %%
%%%%%%%%%%%%%%%%%%%%%%%%%%%%%%%%%%%%%%%%%%%%%%%%%%%%%%%%%%%%%
\frame{\frametitle{Algorithmic implementation: MC-based prediction}
  \setlength{\algomargin}{0.5em}
  \begin{algorithm}[H]
    \SetKwInput{Input}{input}
    \SetKwInput{Output}{output}
    \SetKwInput{Init}{init}
    \SetKwInput{Param}{parameter}
    \Input{a policy $\pi$ to be evaluated}
    \Output{estimate of $\bm{v}_{\mathcal{X}}^{\pi}$ (i.e., value estimate for all states				 $x\in\mathcal{X}$)}
    \Init{$\hat{v}(x)\, \forall \, x\in\mathcal{X}$ arbitrary except $v_0(x)=0$ if $x$ is terminal\newline
      $l(x)\leftarrow$ an empty list for every $x\in\mathcal{X}$}
    \For{$j=1,\ldots,J$ episodes}{
      Generate an episode following $\pi$: $x_{0}, u_{0}, r_{1},\ldots,x_{T_j}, u_{T_j}, r_{T_{j}+1}$ \;
      Set $g \leftarrow 0$\;
      \For{$k=T_j-1, T_j-2, T_j-3,\ldots, 0$ time steps}{
	$g \leftarrow \gamma g + r_{k+1}$\;
	\If{$x_{k}\notin \left\langle x_{0}, x_{1}, \ldots, x_{k-1}\right\rangle$}{
	  Append $g$ to list $l(x_k)$\;
	  $\hat{v}(x_k)\leftarrow \mbox{average}(\,l(x_k)\,)$\;}
      }
    }
    \caption{MC state-value prediction (first visit)}
    \label{algo:MC_state_prediction_first_visit}
  \end{algorithm}
}

%%%%%%%%%%%%%%%%%%%%%%%%%%%%%%%%%%%%%%%%%%%%%%%%%%%%%%%%%%%%%
%% Incremental Implementation%%
%%%%%%%%%%%%%%%%%%%%%%%%%%%%%%%%%%%%%%%%%%%%%%%%%%%%%%%%%%%%%
\frame{\frametitle{Incremental implementation}
  \begin{itemize}
  \item \algoref{algo:MC_state_prediction_first_visit} is inefficient due to large memory demand.
  \item Better: use \hl{incremental / recursive implementation}.\pause
  \item The sample mean $\mu_1,\mu_2,\ldots$ of an arbitrary sequence $g_1, g_2, \ldots$ is:
  \end{itemize}
  \begin{equation}
    \begin{split}
      \mu_J &= \frac{1}{J}\sum_{i=1}^J g_i= \frac{1}{J}\left[g_J + \sum_{i=1}^{J-1} g_i\right]\\
      &= \frac{1}{J}\left[g_J + (J-1) \mu_{J-1}\right] =\mu_{J-1} + \frac{1}{J}\left[g_J-\mu_{J-1}\right].
    \end{split}
    \label{eq:inc_impl_MC_pred}
  \end{equation} \pause
  \begin{itemize}
  \item \hl{If a given decision problem is non-stationary}, using a forgetting factor $\alpha\in\left\{\Re|0<\alpha<1\right\}$ allows for dynamic adaption:
  \end{itemize}
  \begin{equation}
    \mu_J = \mu_{J-1} + \alpha\left[g_J-\mu_{J-1}\right] .
    \label{eq:inc_impl_MC_pred_non_stat}
  \end{equation}
}

%%%%%%%%%%%%%%%%%%%%%%%%%%%%%%%%%%%%%%%%%%%%%%%%%%%%%%%%%%%%%
%% Statistical Properties of MC-Based Prediction (1)%%
%%%%%%%%%%%%%%%%%%%%%%%%%%%%%%%%%%%%%%%%%%%%%%%%%%%%%%%%%%%%%
\frame{\frametitle{Statistical properties of MC-based prediction (1)}
  \onslide<1->{First-time visit MC:
    \begin{itemize}
    \item Each return sample $g_J$ is independent from the others since they were drawn from separate episodes.
    \item One receives \hl{i.i.d.\ data} to estimate $\E{\hat{v}_{\pi}}$ and consequently this \hl{is bias-free}.
    \item The estimate's variance $\Var{\hat{v}_{\pi}}$ drops with $1/n$ ($n$: available samples).
  \end{itemize}}
  \onslide<2->{Every-time visit MC:
    \begin{itemize}
    \item Each return sample $g_J$ is not independent from the others since they might be obtained from same episodes.
    \item One receives \hl{non-i.i.d.} data to estimate $\E{\hat{v}_{\pi}}$ and consequently this \hl{is biased} for any $n<\infty$.
    \item Only in the limit $n\rightarrow \infty$ one receives $\left(v_{\pi}(x)-\E{\hat{v}_{\pi}(x)}\right) \rightarrow 0$.
  \end{itemize}}
  \footnotesize
  \vspace{0.5cm}
  \onslide<1->More information: S. Singh and  R. Sutton, "Reinforcement Learning with Replacing Eligibility Traces", Machine Learning, Vol. 22, pp. 123-158, 1996
}

%%%%%%%%%%%%%%%%%%%%%%%%%%%%%%%%%%%%%%%%%%%%%%%%%%%%%%%%%%%%%
%% Statistical Properties of MC-Based Prediction (2)%%
%%%%%%%%%%%%%%%%%%%%%%%%%%%%%%%%%%%%%%%%%%%%%%%%%%%%%%%%%%%%%
\frame{\frametitle{Statistical properties of MC-based prediction (2)}

  \begin{itemize}
  \item State-value estimates for each state are independent.
  \item One estimate does not rely on the estimate of other states \newline(\hl{no bootstrapping} compared to DP).
  \item Makes MC particularly attractive when one requires state-value knowledge of only one or few states.
    \begin{itemize}
    \item Hence, generating episodes starting from the state of interest.
    \end{itemize}
  \end{itemize}
  \begin{figure}
    \subfloat{
      \includegraphics[height=2.5cm]{fig/lec04/Back_Up_DP.pdf}
    }
    \hspace{1cm}
    \subfloat{
      \includegraphics[height=2.5cm]{fig/lec04/Back_Up_MC.pdf}
    }
    \caption{Back-up diagrams for DP (left) and MC (right) prediction: shallow one-step back-ups with bootstrapping vs. deep back-ups over full epsiodes}
  \end{figure}
}

%%%%%%%%%%%%%%%%%%%%%%%%%%%%%%%%%%%%%%%%%%%%%%%%%%%%%%%%%%%%%
%% MC-Based Predction Example: Forest Tree MDP%%
%%%%%%%%%%%%%%%%%%%%%%%%%%%%%%%%%%%%%%%%%%%%%%%%%%%%%%%%%%%%%
\frame{\frametitle{MC-based prediction example: forest tree MDP (1)}
  Let's reuse the forest tree MDP example with \textit{fifty-fifty policy} and discount factor $\gamma=0.8$
  plus disaster probability $\alpha=0.2$:
  \vspace{0.5cm}
  \begin{figure}
    \includegraphics[height=0.5\textheight]{fig/lec04/Forest_Markov_Decision_Process_State_Value.pdf}
    \caption{Forest MDP with fifty-fifty-policy including state values}
  \end{figure}
}

%%%%%%%%%%%%%%%%%%%%%%%%%%%%%%%%%%%%%%%%%%%%%%%%%%%%%%%%%%%%%
%% MC-Based Predction Example: Forest Tree MDP%%
%%%%%%%%%%%%%%%%%%%%%%%%%%%%%%%%%%%%%%%%%%%%%%%%%%%%%%%%%%%%%
\frame{\frametitle{MC-based prediction example: forest tree MDP (2)}
  \begin{figure}
    \includegraphics[height=0.65\textheight]{fig/lec04/Forest_Tree_MC_Value_Prediction.pdf}
    \caption{State-value estimate of forest tree MDP initial state using MC-based prediction over the number of episodes being evaluated (mean and standard deviation are calculated based on 2000 independent runs)}
    \label{fig:Forest_Tree_MC_Value_Prediction}
  \end{figure}
}

%%%%%%%%%%%%%%%%%%%%%%%%%%%%%%%%%%%%%%%%%%%%%%%%%%%%%%%%%%%%%
%% MC Estimation of Action Values %%
%%%%%%%%%%%%%%%%%%%%%%%%%%%%%%%%%%%%%%%%%%%%%%%%%%%%%%%%%%%%%
\frame{\frametitle{MC estimation of action values}
  Is a \hl{model available} (i.e., tuple $\left\langle\mathcal{X}, \mathcal{U}, \bm{\mathcal{P}}, \mathcal{R}, \gamma \right\rangle$)?
  \begin{itemize}
  \item \hl{Yes}: state values plus one-step prediction deliver optimal policy.
  \item \hl{No}: action values are very useful to directly obtain optimal choices.
  \item Recap policy improvement from last lecture.
  \end{itemize}\pause
  \hl{Estimating $q{_\pi}(x, u)$} using MC approach:
  \begin{itemize}
  \item Analog to state values summarized in \algoref{algo:MC_state_prediction_first_visit}.
  \item Only small extension: a visit refers to a state-action pair $(x, u)$.
  \item First-visit and every-visit variants exist.
  \end{itemize}\pause
  Possible problem when following a deterministic policy $\pi$:
  \begin{itemize}
  \item Certain state-action pairs $(x, u)$ are never visited.
  \item Missing degree of exploration.\pause
  \item Workaround: \hl{exploring starts}, i.e., starting episodes in random state-action pairs $(x, u)$ and thereafter following  $\pi$.
  \end{itemize}
}

%%%%%%%%%%%%%%%%%%%%%%%%%%%%%%%%%%%%%%%%%%%%%%%%%%%%%%%%%%%%%%%%%%
\section{Basic Monte Carlo control}
%%%%%%%%%%%%%%%%%%%%%%%%%%%%%%%%%%%%%%%%%%%%%%%%%%%%%%%%%%%%%%%%%%
\begin{frame}
  \frametitle{Table of contents}
  \tableofcontents[currentsection]
\end{frame}

%%%%%%%%%%%%%%%%%%%%%%%%%%%%%%%%%%%%%%%%%%%%%%%%%%%%%%%%%%%%%
%% Applying Generalized Policy Iteration (GPI) to MC Control %%
%%%%%%%%%%%%%%%%%%%%%%%%%%%%%%%%%%%%%%%%%%%%%%%%%%%%%%%%%%%%%
\frame{\frametitle{Applying generalized policy iteration (GPI) to MC control}
  \onslide<1->{GPI concept is directly applied to MC framework using action values:
    \begin{equation}
      \label{eq:GPI_MC}
      \pi_0 \rightarrow \hat{q}_{\pi_0} \rightarrow \pi_1 \rightarrow \hat{q}_{\pi_1} \rightarrow \cdots \pi^* \rightarrow \hat{q}_{\pi^*} \, .
  \end{equation}}
  \begin{itemize}
    \onslide<2->{\item Degree of freedom: Choose number of episodes to approximate $\hat{q}_{\pi_i}$.}
    \onslide<3->{\item Policy improvement is done by greedy choices:
      \begin{equation}
	\pi(x)= \argmax_{u} q(x, u) \quad \forall x\in\mathcal{X}\, .
    \end{equation}}
  \end{itemize}
  \onslide<1->{\begin{figure}
      \begin{minipage}[c]{0.25\textwidth}
        \includegraphics[height=0.4\textheight]{fig/lec04/MC_GPI.pdf}
      \end{minipage}
      \begin{minipage}[c]{0.45\textwidth}
        \caption{Transferring GPI to MC-based control (source: R. Sutton and G. Barto, Reinforcement learning: an introduction, 2018, \href{https://creativecommons.org/licenses/by-nc-nd/2.0/}{CC BY-NC-ND 2.0})}
	\label{fig:GPI_MC}
      \end{minipage}
  \end{figure}}
}

%%%%%%%%%%%%%%%%%%%%%%%%%%%%%%%%%%%%%%%%%%%%%%%%%%%%%%%%%%%%%
%% Policy Improvement Theoreom %%
%%%%%%%%%%%%%%%%%%%%%%%%%%%%%%%%%%%%%%%%%%%%%%%%%%%%%%%%%%%%%
\frame{\frametitle{Policy improvement theorem}
  Assuming that one is operating in an \hl{unknown MDP}, the policy improvement theorem \theoref{theo:Policy_improvement} is still valid for MC-based control:
  \begin{block}{Policy improvement for MC-based control}
    \begin{equation}
      \begin{split}
	q_{\pi_i}(x,\pi_{i+1}(x)) &= q_{\pi_i}(x,\argmax_{u} q_{\pi_i}(x,u)),\\
	&= \max_{u} q_{\pi_i}(x,u),\\
	&\geq q_{\pi_i}(x,\pi_{i}(x)),\\
	&\geq v_{\pi_i}(x).
      \end{split}
    \end{equation}
  \end{block}
  \begin{itemize}
  \item Each $\pi_{i+1}$ is uniformly better or just as good (if optimal) as $\pi_{i}$.
  \item Assumption: All state-action pairs are evaluated due to sufficient exploration.
    \begin{itemize}
    \item For example using exploring starts.
    \end{itemize}
  \end{itemize}
}

%%%%%%%%%%%%%%%%%%%%%%%%%%%%%%%%%%%%%%%%%%%%%%%%%%%%%%%%%%%%%
%% Algorithmic Implementation: MC-Based Control Using Exploring Start %%
%%%%%%%%%%%%%%%%%%%%%%%%%%%%%%%%%%%%%%%%%%%%%%%%%%%%%%%%%%%%%
\frame{\frametitle{Algorithmic implementation: MC-based control}
  \setlength{\algomargin}{0.5em}
  \begin{algorithm}[H]
    \footnotesize
    \SetKwInput{Input}{input}
    \SetKwInput{Output}{output}
    \SetKwInput{Init}{init}
    \SetKwInput{Param}{parameter}
    \Output{Optimal deterministic policy $\pi^*$}
    \Init{$\pi_{i=0}(x)\in\mathcal{U}$ arbitrarily $\forall x\in\mathcal{X}$\newline
      $\hat{q}(x,u)$ arbitrarily $\forall \, \left\{x\in\mathcal{X}, u\in\mathcal{U}\right\}$\newline
      $n(x,u)\leftarrow$ an empty list for state-action visits $\forall \, \left\{x\in\mathcal{X}, u\in\mathcal{U}\right\}$}
    \Repeat{$\pi_{i+1} = \pi_{i}$}{
      $i\leftarrow i+1$ \;
      Choose $\left\{x_{0}, u_{0}\right\}$ randomly such that all pairs have probability $>$ 0 \;
      Generate an episode from $\left\{x_{0}, u_{0}\right\}$ following $\pi_{i}$ until termination step $T_i$\;
      Set $g \leftarrow 0$\;
      \For{$k=T_i-1, T_i-2, T_i-3,\ldots, 0$ time steps}{
	$g \leftarrow \gamma g + r_{k+1}$\;
	\If{$\left\{x_{k}, u_{k}\right\}\notin \left\langle \left\{x_{0}, u_{0}\right\}, \ldots, \left\{x_{k-1}, u_{k-1}\right\}\right\rangle$}{
	  $n(x_k,u_k)\leftarrow n(x_k,u_k)+1$\;
	  $\hat{q}(x_k,u_k)\leftarrow \hat{q}(x_k,u_k) + 1/n(x_k,u_k)\cdot(g-\hat{q}(x_k,u_k))$\;
	  $\pi_{i}(x_k)\leftarrow\argmax_{u}\,\, \hat{q}(x_k,u)$\;
	}
      }
    }
    \caption{MC-based control using exploring starts (first visit)}
    \label{algo:MC_ES}
  \end{algorithm}
}

%%%%%%%%%%%%%%%%%%%%%%%%%%%%%%%%%%%%%%%%%%%%%%%%%%%%%%%%%%%%%%%%%%
\section{Extensions to Monte Carlo on-policy control}
%%%%%%%%%%%%%%%%%%%%%%%%%%%%%%%%%%%%%%%%%%%%%%%%%%%%%%%%%%%%%%%%%%
\begin{frame}
  \frametitle{Table of contents}
  \tableofcontents[currentsection]
\end{frame}

%%%%%%%%%%%%%%%%%%%%%%%%%%%%%%%%%%%%%%%%%%%%%%%%%%%%%%%%%%%%%
%% Off- and On-Policy Learning %%
%%%%%%%%%%%%%%%%%%%%%%%%%%%%%%%%%%%%%%%%%%%%%%%%%%%%%%%%%%%%%
\frame{\frametitle{Off- and on-policy learning}
  \begin{itemize}
  \item \hl{On-policy learning}
    \begin{itemize}
    \item Evaluate or improve the policy used to make decisions.
    \item Agent picks own actions.
    \item Exploring starts (ES) is an on-policy method example.
    \item However: ES is a restrictive assumption and not always applicable \newline (in some cases the starting state-action pair cannot be choosen freely).
    \end{itemize}\pause
    \vspace{1cm}
  \item \hl{Off-policy learning}
    \begin{itemize}
    \item Evaluate or improve a policy different from that used to generate data.
    \item Agent cannot apply own actions.
    \item Will be focused in the next sections.
    \end{itemize}
  \end{itemize}
}

%%%%%%%%%%%%%%%%%%%%%%%%%%%%%%%%%%%%%%%%%%%%%%%%%%%%%%%%%%%%%
%% Epsilon-Greedy as an On-Policy Alternative %%
%%%%%%%%%%%%%%%%%%%%%%%%%%%%%%%%%%%%%%%%%%%%%%%%%%%%%%%%%%%%%
\frame{\frametitle{$\varepsilon$-greedy as an on-policy alternative}
  \begin{itemize}
  \item Exploration requirement:
    \begin{itemize}
    \item Visit all state-action pairs with probability:
      \begin{equation}
	\pi(u|x) > 0 \quad \forall\, \left\{x\in\mathcal{X}, u\in\mathcal{U}\right\}\, .
      \end{equation}
    \item Policies with this characteristic are called: \hl{soft}.
    \item Level of exploration can be tuned during the learning process.
    \end{itemize}\pause
    \vspace{0.5cm}
  \item \hl{$\varepsilon$-greedy on-policy learning}
    \begin{itemize}
    \item With probability $\varepsilon$ the agent's choice (i.e., the policy output) is overwritten with a random action.
    \item Probability of all non-greedy actions: \begin{equation}\varepsilon/|\mathcal{U}|\,.\end{equation}
    \item Probability of the greedy action: \begin{equation}1-\varepsilon+\varepsilon/|\mathcal{U}|\,.\end{equation}
    \item Above, $|\mathcal{U}|$ is the cardinality of the action space.
    \end{itemize}
  \end{itemize}
}

%%%%%%%%%%%%%%%%%%%%%%%%%%%%%%%%%%%%%%%%%%%%%%%%%%%%%%%%%%%%%
%% Algorithmic implementation $\varepsilon$-Greedy MC-Control %%
%%%%%%%%%%%%%%%%%%%%%%%%%%%%%%%%%%%%%%%%%%%%%%%%%%%%%%%%%%%%%
\frame{\frametitle{Algorithmic implementation $\varepsilon$-greedy MC-control}
  \setlength{\algomargin}{0.5em}
  \begin{algorithm}[H]
    \footnotesize
    \SetKwInput{Input}{input}
    \SetKwInput{Output}{output}
    \SetKwInput{Init}{init}
    \SetKwInput{Param}{parameter}
    \Output{Optimal $\varepsilon$-greedy policy $\pi^*(u|x)$, \hspace{0.5cm}\textbf{parameter:} $\varepsilon\in\left\{\Re|0<\varepsilon<<1\right\}$}%\Param{}
    \Init{$\pi_{i=0}(u|x)$ arbitrarily soft $\forall \, \left\{x\in\mathcal{X}, u\in\mathcal{U}\right\}$\newline
      $\hat{q}(x,u)$ arbitrarily $\forall \, \left\{x\in\mathcal{X}, u\in\mathcal{U}\right\}$\newline
      $n(x,u)\leftarrow$ an empty list counting state-action visits $\forall \, \left\{x\in\mathcal{X}, u\in\mathcal{U}\right\}$}
    \Repeat{$\pi_{i+1} = \pi_{i}$}{
      Generate an episode following $\pi_i$: $x_{0}, u_{0}, r_{1},\ldots,x_{T_j}, u_{T_j}, r_{T_{j+1}}$ \;
      $i\leftarrow i+1$ \;
      Set $g \leftarrow 0$\;
      \For{$k=T_i-1, T_i-2, T_i-3,\ldots, 0$ time steps}{
	$g \leftarrow \gamma g + r_{k+1}$\;
	\If{$\left\{x_{k}, u_{k}\right\}\notin \left\langle \left\{x_{0}, u_{0}\right\}, \ldots, \left\{x_{k-1}, u_{k-1}\right\}\right\rangle$}{
	  $n(x_k,u_k)\leftarrow n(x_k,u_k)+1$\;
	  $\hat{q}(x_k,u_k)\leftarrow \hat{q}(x_k,u_k) + 1/n(x_k,u_k)\cdot(g-\hat{q}(x_k,u_k))$\;
	  $\tilde{u}\leftarrow\argmax_{u}\,\, \hat{q}(x_k,u)$\;
	  $\pi_i(u|x_k)\leftarrow\begin{cases}1-\varepsilon+\varepsilon/|\mathcal{U}|, \quad u=\tilde{u}\\ \varepsilon/|\mathcal{U}|, \quad u\neq\tilde{u}\end{cases}$\;
	}
      }
    }
    \caption{MC-based control using $\varepsilon$-greedy approach}
    \label{algo:MC_eps_greedy}
  \end{algorithm}
}

%%%%%%%%%%%%%%%%%%%%%%%%%%%%%%%%%%%%%%%%%%%%%%%%%%%%%%%%%%%%%
%% $\varepsilon$-Greedy Policy Improvement (1)%%
%%%%%%%%%%%%%%%%%%%%%%%%%%%%%%%%%%%%%%%%%%%%%%%%%%%%%%%%%%%%%
\frame{\frametitle{$\varepsilon$-greedy policy improvement (1)}
  \begin{theo}{Policy improvement for $\varepsilon$-greedy policy}{policy_improv_eps_greedy}
    Given an MDP, for any $\varepsilon$-greedy policy $\pi$ the $\varepsilon$-greedy policy $\pi'$ with respect to $q_\pi$ is an improvement, i.e., $v_{\pi'} > v_{\pi} \quad \forall x\in\mathcal{X}$.
  \end{theo}\pause
  Small proof:
  \small
  \begin{equation}
    \begin{split}
      q_{\pi}(x,\pi'(x)) &= \sum_{u}\pi'(u|x)q_{\pi}(x,u),\\
      &= \frac{\varepsilon}{|\mathcal{U}|}\sum_{u}q_{\pi}(x,u) + (1-\varepsilon)\max_{u}q_{\pi}(x,u),\\
      &\geq \frac{\varepsilon}{|\mathcal{U}|}\sum_{u}q_{\pi}(x,u) + (1-\varepsilon)\sum_{u}\frac{\pi(u|x)-\frac{\varepsilon}{|\mathcal{U}|}}{1-\varepsilon}q_{\pi}(x,u).
    \end{split}
  \end{equation}
  \normalsize
  In the inequality line, the second term is the weighted sum over action values given an $\varepsilon$-greedy policy. This weighted sum will be always smaller or equal than  $\max_{u}q_{\pi}(x,u)$.
}

%%%%%%%%%%%%%%%%%%%%%%%%%%%%%%%%%%%%%%%%%%%%%%%%%%%%%%%%%%%%%
%% $\varepsilon$-Greedy Policy Improvement (2)%%
%%%%%%%%%%%%%%%%%%%%%%%%%%%%%%%%%%%%%%%%%%%%%%%%%%%%%%%%%%%%%
\frame{\frametitle{$\varepsilon$-greedy policy improvement (2)}
  Continuation:
  \small
  \begin{equation}
    \begin{split}
      q_{\pi}(x,\pi'(x)) &\geq \frac{\varepsilon}{|\mathcal{U}|}\sum_{u}q_{\pi}(x,u) + (1-\varepsilon)\sum_{u}\frac{\pi(u|x)-\frac{\varepsilon}{|\mathcal{U}|}}{1-\varepsilon}q_{\pi}(x,u),\\
      &= \frac{\varepsilon}{|\mathcal{U}|}\sum_{u}\left(q_{\pi}(x,u) - q_{\pi}(x,u)\right) + \sum_{u}\pi(u|x)q_{\pi}(x,u),\\
      &=\sum_{u}\pi(u|x)q_{\pi}(x,u),\\
      &=v_{\pi}(x).
    \end{split}
  \end{equation}
  \normalsize\pause
  \begin{itemize}
  \item Policy improvement theorem is still valid when comparing $\varepsilon$-greedy policies against each other.
  \item But: There might be a non-$\varepsilon$-greedy policy which is better.
  \end{itemize}
}

%%%%%%%%%%%%%%%%%%%%%%%%%%%%%%%%%%%%%%%%%%%%%%%%%%%%%%%%%%%%%
%% MC-Based Control Example: Forest Tree MDP (1)%%
%%%%%%%%%%%%%%%%%%%%%%%%%%%%%%%%%%%%%%%%%%%%%%%%%%%%%%%%%%%%%
\frame{\frametitle{MC-based control example: forest tree MDP (1)}
  \begin{figure}
    \includegraphics[height=0.65\textheight]{fig/lec04/Forest_Tree_MC_Control_eps_02.pdf}
    \caption{Different estimates of forest tree MDP ($\alpha=0.2, \gamma=0.8$) using MC control with \hl{$\varepsilon=0.2$} over the number of episodes. Mean is red and standard deviation is light blue, both calculated based on 2000 independent uns.}
    \label{fig:Forest_Tree_MC_Control_eps_02}
  \end{figure}
}

%%%%%%%%%%%%%%%%%%%%%%%%%%%%%%%%%%%%%%%%%%%%%%%%%%%%%%%%%%%%%
%% MC-Based Control Example: Forest Tree MDP (2)%%
%%%%%%%%%%%%%%%%%%%%%%%%%%%%%%%%%%%%%%%%%%%%%%%%%%%%%%%%%%%%%
\frame{\frametitle{MC-based control example: forest tree MDP (2)}
  \begin{figure}
    \includegraphics[height=0.65\textheight]{fig/lec04/Forest_Tree_MC_Control_eps_005.pdf}
    \caption{Different estimates of forest tree MDP ($\alpha=0.2, \gamma=0.8$) using MC control with \hl{$\varepsilon=0.05$} over the number of episodes. Mean is red and standard deviation is light blue, both calculated based on 2000 independent runs.}
    \label{fig:Forest_Tree_MC_Control_eps_005}
  \end{figure}
}

%%%%%%%%%%%%%%%%%%%%%%%%%%%%%%%%%%%%%%%%%%%%%%%%%%%%%%%%%%%%%
%% MC-Based Control Example: Forest Tree MDP (3)%%
%%%%%%%%%%%%%%%%%%%%%%%%%%%%%%%%%%%%%%%%%%%%%%%%%%%%%%%%%%%%%
\frame{\frametitle{MC-based control example: forest tree MDP (3)}
  Observations on forest tree MDP with $\varepsilon$-greedy MC-based control:
  \begin{itemize}
  \item Rather slow convergence rate: quite a number of episodes is required. \pause
  \item Significant uncertainty present in a single sequence. \pause
  \item Later states are less often visited and, therefore, more uncertain. \pause
  \item Exploration is controlled by $\varepsilon$: in a totally greedy policy the state $x=3$ is not visited at all (cf. \figref{fig:Forest_Markov_Decision_Process_Optimal_Action_Value}). With $\varepsilon$-greedy this state is visited occasionally. \pause
  \item Nevertheless, the above figures highlight that MC-based control for the forest tree MDP tend towards the optimal policies discovered by dynamic programming (cf. \tabref{tab:Forest_tree_value_iteration}).
  \end{itemize}
}

%%%%%%%%%%%%%%%%%%%%%%%%%%%%%%%%%%%%%%%%%%%%%%%%%%%%%%%%%%%%%
%% Greedy in the Limit with Infinite Exploration (GLIE) %%
%%%%%%%%%%%%%%%%%%%%%%%%%%%%%%%%%%%%%%%%%%%%%%%%%%%%%%%%%%%%%
\frame{\frametitle{Greedy in the limit with infinite exploration (GLIE)}
  \begin{defi}{Greedy in the limit with infinite exploration (GLIE)}{GLIE}
    A learning policy $\pi$ is called GLIE if it satisfies the following two properties:
    \begin{itemize}
    \item If a state is visited infinitely often, then each action is chosen infinitely often:
      \begin{equation}
	\lim_{i\rightarrow\infty} \pi_i(u|x)=1 \quad \forall \, \left\{x\in\mathcal{X}, u\in\mathcal{U}\right\}\, .
      \end{equation}
    \item In the limit ($i\rightarrow \infty$) the learning policy is greedy with respect to the learned action value:
      \begin{equation}
	\lim_{i\rightarrow\infty} \pi_i(u|x)=\pi(x)= \argmax_{u} q(x, u) \quad \forall x\in\mathcal{X} \, .
      \end{equation}
    \end{itemize}
  \end{defi}
}

%%%%%%%%%%%%%%%%%%%%%%%%%%%%%%%%%%%%%%%%%%%%%%%%%%%%%%%%%%%%%
%% GlIE Monte-Carlo Control %%
%%%%%%%%%%%%%%%%%%%%%%%%%%%%%%%%%%%%%%%%%%%%%%%%%%%%%%%%%%%%%
\frame{\frametitle{GLIE Monte Carlo control}
  \begin{theo}{Optimal decision using MC-control with $\varepsilon$-greedy}{GLIE_MC}
    MC-based control using $\varepsilon$-greedy exploration is GLIE, if $\varepsilon$ is decreased at rate
    \begin{equation}
      \varepsilon_i = \frac{1}{i}
    \end{equation}
    with $i$ being the increasing episode index. In this case,
    \begin{equation}
      \hat{q}(x, u) = q^*(x, u)
    \end{equation}
    follows.
  \end{theo}
  \pause
  Remarks:
  \begin{itemize}
  \item Limited feasibility: infinite number of episodes required.
  \item $\varepsilon$-greedy is an undirected and unmonitored random exploration strategy. Can that be the most efficient way of learning?
  \end{itemize}
}

%%%%%%%%%%%%%%%%%%%%%%%%%%%%%%%%%%%%%%%%%%%%%%%%%%%%%%%%%%%%%%%%%%
\section{Monte Carlo off-policy prediction and control}
%%%%%%%%%%%%%%%%%%%%%%%%%%%%%%%%%%%%%%%%%%%%%%%%%%%%%%%%%%%%%%%%%%
\begin{frame}
  \frametitle{Table of contents}
  \tableofcontents[currentsection]
\end{frame}

%%%%%%%%%%%%%%%%%%%%%%%%%%%%%%%%%%%%%%%%%%%%%%%%%%%%%%%%%%%%%
%% Off-Policy Learning Background %%
%%%%%%%%%%%%%%%%%%%%%%%%%%%%%%%%%%%%%%%%%%%%%%%%%%%%%%%%%%%%%
\frame{\frametitle{Off-policy learning background}
  Drawback of on-policy learning:
  \begin{itemize}
  \item Only a compromise: comes with inherent exploration but at the cost of learning action values for a \hl{near-optimal policy}.
  \end{itemize}
  \vspace{0.75cm}\pause
  Idea off-policy learning:
  \begin{itemize}
  \item Use two separated policies:
    \begin{itemize}
    \item \hl{Behavior policy} $b(u|x)$: explores in order to generate experience.
    \item \hl{Target policy} $\pi(u|x)$: learns from that experience to become the optimal policy.
    \end{itemize}\pause
  \item Use cases:
    \begin{itemize}
    \item Learn from observing humans or other agents/controllers. \pause
    \item Re-use experience generated from old policies ($\pi_0,\pi_1,\ldots$). \pause
    \item Learn about multiple policies while following one policy.
    \end{itemize}
  \end{itemize}
}

%%%%%%%%%%%%%%%%%%%%%%%%%%%%%%%%%%%%%%%%%%%%%%%%%%%%%%%%%%%%%
%% Off-Policy Prediction Problem Statement %%
%%%%%%%%%%%%%%%%%%%%%%%%%%%%%%%%%%%%%%%%%%%%%%%%%%%%%%%%%%%%%
\frame{\frametitle{Off-policy prediction problem statement}
  \begin{block}{MC off-policy prediction problem statement}
    \begin{itemize}
    \item Estimate $v_\pi$ and/or $q_\pi$ while following $b(u|x)$.
    \item Both policies are considered fixed (prediction assumption).
    \end{itemize}
  \end{block}
  \vspace{0.75cm}\pause
  Requirement:
  \begin{itemize}
  \item \hl{Coverage}: Every action taken under $\pi$ must be (at least occasionally) taken under $b$, too. Hence, it follows:
    \begin{equation}
      \pi(u|x) > 0 \Rightarrow b(u|x) > 0 \quad \forall \, \left\{x\in\mathcal{X}, u\in\mathcal{U}\right\} .
    \end{equation}\pause
  \item Consequences from that:
    \begin{itemize}
    \item In any state $b$ is not identical to $\pi$, $b$ must be stochastic.
    \item Nevertheless: $\pi$ might be deterministic (e.g., control applications) or stochastic.
    \end{itemize}
  \end{itemize}
}

%%%%%%%%%%%%%%%%%%%%%%%%%%%%%%%%%%%%%%%%%%%%%%%%%%%%%%%%%%%%%
%% Importance Sampling %%
%%%%%%%%%%%%%%%%%%%%%%%%%%%%%%%%%%%%%%%%%%%%%%%%%%%%%%%%%%%%%
\frame{\frametitle{Importance sampling}
  Probability of observing a certain trajectory on random variables $U_k, X_{k+1}, U_{k+1}, \ldots, X_{T}$ starting in $X_{k}$ while following $\pi$:
  \begin{equation}
    \begin{split}
      &\Pb{U_k, X_{k+1}, U_{k+1}, \ldots , X_{T}| X_{k}, \pi}=\pi(U_k|X_{k})p(X_{k+1}|X_{k}, U_{k})\pi(U_{k+1}|X_{k+1})\cdots\,,\\
      &=\prod_k^{T-1} \pi(U_k|X_{k})p(X_{k+1}|X_{k}, U_{k}).
    \end{split}
  \end{equation}
  Above $p$ is the state-transition probability (cf. \defref{defi:Markov_decision_process}).\pause
  \begin{defi}{Importance sampling ratio}{import_sampl_ratio}
    The relative probability of a trajectory under the target and behavior policy, the importance sampling ratio, from sample step $k$ to $T$ is:
    \begin{equation}
      \rho_{k:T}=\frac{\prod_k^{T-1} \pi(U_k|X_{k})p(X_{k+1}|X_{k}, U_{k})}{\prod_k^{T-1} b(U_k|X_{k})p(X_{k+1}|X_{k}, U_{k})}=\frac{\prod_k^{T-1} \pi(U_k|X_{k})}{\prod_k^{T-1} b(U_k|X_{k})} .
      \label{eq:import_sampl_ratio}
    \end{equation}
  \end{defi}
}

%%%%%%%%%%%%%%%%%%%%%%%%%%%%%%%%%%%%%%%%%%%%%%%%%%%%%%%%%%%%%
%% Importance Sampling for Monte Carlo Prediction %%
%%%%%%%%%%%%%%%%%%%%%%%%%%%%%%%%%%%%%%%%%%%%%%%%%%%%%%%%%%%%%
\frame{\frametitle{Importance sampling for Monte Carlo prediction}
  \begin{defi}{State-value estimation via Monte Carlo importance sampling}{MC_ord_import_sampl}
    Estimating the state value $v_\pi$ following a behavior policy $b$ using (ordinary) importance sampling (OIS) results in scaling and averaging the sampled returns by the importance sampling ratio per episode:
    \begin{equation}
      \hat{v}_\pi(x_k)=\frac{\sum_{k\in\mathcal{T}(x_k)}\rho_{k:T(k)}g_k}{|\mathcal{T}(x_k)|}.
      \label{eq:OIS}
    \end{equation}
  \end{defi}
  Notation remark:
  \begin{itemize}
  \item $\mathcal{T}(x_k)$: set of all time steps in which the state $x_k$ is visited.
  \item $T(k)$: Termination of a specific episode starting from $k$.
  \end{itemize}\pause
  General remark:
  \begin{itemize}
  \item From \eqref{eq:import_sampl_ratio} it can be seen that $\hat{v}$ is bias-free (first-visit assumption).
  \item However, if $\rho$ is large (distinctly different policies) the estimate's variance is large (i.e., uncertain for small numbers of samples).
  \end{itemize}
}


%%%%%%%%%%%%%%%%%%%%%%%%%%%%%%%%%%%%%%%%%%%%%%%%%%%%%%%%%%%%%
%% Off-Policy Monte Carlo Control: Introduction %%
%%%%%%%%%%%%%%%%%%%%%%%%%%%%%%%%%%%%%%%%%%%%%%%%%%%%%%%%%%%%%
\frame{\frametitle{Off-policy Monte Carlo control}
  Just put everything together:
  \begin{itemize}
  \item MC-based control utilizing GPI (cf. \figref{fig:GPI_MC}),
  \item Off-policy learning based on importance sampling (or variants like weighted importance sampling, cf. Barto/Sutton book chapter 5.5).
  \end{itemize}
  \vspace{1cm} \pause
  Requirement for off-policy MC-based control:
  \begin{itemize}
  \item \hl{Coverage}: behavior policy $b$  has nonzero probability of selecting actions that might be taken by the target policy $\pi$.
  \item Consequence: behavior policy $b$ is \hl{soft} (e.g., $\varepsilon$-soft).
  \end{itemize}
}

%%%%%%%%%%%%%%%%%%%%%%%%%%%%%%%%%%%%%%%%%%%%%%%%%%%%%%%%%%%%%
%% Summary %%
%%%%%%%%%%%%%%%%%%%%%%%%%%%%%%%%%%%%%%%%%%%%%%%%%%%%%%%%%%%%%
\begin{frame}
  \frametitle{Summary: what you've learned today}
  \begin{itemize}
  \item MC methods allow model-free learning of value functions and optimal policies from experience in the form of sampled episodes.
  \item Using deep back-ups over full episodes, MC is largely based on averaging returns. \pause
  \item MC-based control reuses generalized policy iteration (GPI), i.e., mixing policy evaluation and improvement. \pause
  \item Maintaining sufficient exploration is important:
    \begin{itemize}
    \item Exploring starts: not feasible in all applications but simple.
    \item On-policy $\varepsilon$-greedy learning: trade-off between optimality and exploration cannot be resolved easily.
    \item Off-policy learning: agent learns about a (possibly deterministic) target policy from an exploratory, soft behavior policy.
    \end{itemize}\pause
  \item Importance sampling transforms expectations from the behavior to the target policy.
    \begin{itemize}
    \item This estimation task comes with a bias-variance-dilemma.
    \item Slow learning can result from ineffective experience usage in MC methods.
    \end{itemize}
  \end{itemize}
\end{frame}
